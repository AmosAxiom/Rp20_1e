\chapter{Combat}
Some of you just jumped straight to this section, didn’t you? Oh not quite? You skimmed through the rest of it? Relax, that’s forgivable. Combat is a large part of most roleplaying systems, and Rp20 is no exception. Every attribute has an impact on combat in some form or another.

Combat in Rp20 follows many other roleplaying systems by occurring in 6 second (time in-game, not in real life) intervals called ``rounds.'' At the start of an encounter, all involved characters roll for initiative. Each round, the character with the highest initiative decides what their actions for the round will be then the character with the next highest initiative, and so on.

Some encounters may have a surprise round, where one party of characters ambushes another party of characters. When this is the case, the characters being ambushed do not roll for initiative until all the characters who are doing the ambush have already performed or prepared their actions for the round. Then they may also decide upon their actions for the round.

When there is a tie for initiative, a coinflip decides who goes first.

\section{Taking an Action}
Taking an action typically consists of rolling for a skill, moving your character, or making an attack. Some skills or class abilities may consume multiple actions to perform.

Talking typically consumes no actions at all, regardless of how much your character speaks. However, if  you talk for more than 30 seconds of real-life time without taking a single action, your GM is allowed to rule that your character has run out of breath and is now unconscious. The only situation in which talking consumes an action is if it is done with the intent to perform a Diplomacy or Bluff check.

Every character has a certain number of actions they may take in a round. This number of actions is equal to (Wis/2) + 3 rounded down (not towards zero), but not more than 6. In other words, a character with a Wisdom of 10 can take 3 actions per round, and for every 4 Wisdom above that, they can take one additional action per round. Characters with a Wisdom of 22 or more take 6 actions, and further increases in Wisdom do not increase the number of actions they may take. However certain classes may still award actions beyond the normal limit of 6 for investing levels in them.

Note that a character with a Wisdom less than 10 begins to have less than 3 actions per round. A character with a Wisdom of 0 or 1 can take no actions per round, and is incapacitated.

\section{Movement}
Every time you take a movement action, you may move your character up to a certain distance. Typically, Rp20 is played on a square grid, where each square is 2ftx2ft. If played on other grids, it is recommended to make each grid space have an incircle diameter of 2ft (each grid space can exactly circumscribe a circle with a diameter of 2ft). The size of other grid shapes is listed below:
\begin{description}
\item[Hex grid] - The edge length of each hexagon for a regular hex grid should be approximately 1.15 ft.
\item[Triangular grid] - An equilateral triangle grid should have an edge length of approximately 6.9 ft. (Yes you read that right, I promise the areas and distances will work out).
\end{description}
Each character stands at the center of the space they are in, and by default, their limbs may reach to the end of all spaces adjacent to them. There are some situations such as naval or space combat where the size of each square may be larger. A character in a space adjacent to another character in Rp20 is standing uncomfortably close, and characters limbs and other body parts poke out of the space they are in and into adjacent spaces. Keep that in mind for both roleplaying and combat.

The number of spaces your character can move per move action is equal to (Str/2) + 3 rounded down (not towards zero), but not more than 6. In other words, a character with a Strength of 10 can move up to 3 squares per turn, and for every 4 Strength above that, they can move one additional space. Characters with a Strength of 22 or more may move 6 spaces per move action, but further investments in Strength may not increase move beyond that.

Note that a character with a Strength less than 10 starts to move less than 3 squares per action. A character with a Strength of 0 or 1 cannot move under their own power during a round by using a normal movement action.
There are a few other skills and actions you can take that result in movement of some kind such as dodging and jumping, and they are described in the ``Special Skills'' section.

\section{Taking Damage}
When you take damage, you subtract the amount of damage from your current hitpoints. All characters actually reduce damage from each source of damage by a number equal to their Con modifier. Note that this means low Con characters will take slightly more damage from damage sources if their modifier is negative. If your character is affected by a poison or something that reduces constitution, they will take more damage.

When your character reaches 0 hitpoints, they are incapacitated, not killed. By default, they may still talk, but they have 0 actions per round. Every subsequent round, they lose one additional hitpoint until they reach -10 hitpoints, at which point they die. To prevent this, a character may roll $Fortitude + (2 \times hitpoints)$ at the beginning of each round. A check above 10 or a critical success stabilizes the character, which prevents them from losing one hitpoint every round. A stabilized character that is not unconscious due to any other factors has one action per round.

Some damage sources may do non-lethal damage. Non-lethal damage cannot reduce your character’s health below -9, and any character that falls to 0 hp or below is automatically stabilized.

Falling damage is a special source of damage which your character takes whenever they fall or are propelled at high velocity towards a solid object. For every 10 feet that someone falls, they roll an additional 1d6 of falling damage. Falling damage can be negated by rolling a Tumble/Roll check. For every 10 you receive on a tumble check, you remove 1d6 of falling damage.

Falling damage is related to knockback. Certain weapons have knockback, and weapons that do more than 10 feet of knockback, if the full knockback distance is not free of obstacles, characters will instead take falling damage as if they had fallen the remaining knockback distance.

\section{Attacking}
When you make an attack on another entity, you first roll to see if you land the attack Your attack must make it past any protection and accurately predict where the target will be (quantified by the target’s armor class). To make an attack roll for most weapons, you roll 1d20+Dex+Weapon Skill. For strength based melee weapons, you may roll 1d20+Str+Weapon Skill. For barehanded melee attacks, you may choose whether to roll 1d20+Dex+Weapon Skill or 1d20+Str+Weapon Skill.

The most basic form of attack is a barehanded melee attack. By default, characters can attack any character in the 8 squares surrounding their square with a barehanded melee attack. Some melee weapons may have longer range, and allow you to attack farther away. Any melee weapon with a range of 2 squares away from your character or less is considered short-range, and other melee weapons like flails or whips are considered long range.

A barehanded melee attack by default does STRd4 damage. For instance, a character with a Str of +3 will do 3d4 damage. Melee weapons often do this base damage with some additional modifier on top.

Melee weapons can be one of two types. They may be strength weapons, or they may be finesse weapons. Strength weapons usually do damage based on Strength in some way. Finesse weapons do damage based on Dex in some way. The damage for weapons are discussed in the “Items, Loot, and Crafting” session.

There are two attacks that actually use Str instead of Dex for attack bonus. Barehanded short-range melee attacks, and strength based melee.

Ranged weapons are either projectile or beam weapons. There is no difference in the way damage is dealt between the two, but they require different weapon proficiencies.

Multi-wielding weapons is also allowed, and allows you to take a single action to attack with all the weapons you are wielding. However you take significant penalties to attack checks the more weapons you are wielding. By default, the penalty starts at -7 for two weapons, goes up to -15 for three weapons, and wielding more weapons than that simultaneously takes a penalty equal to the sum of previous two weapon penalties. The penalties by default exist regardless of the number of limbs or manipulative extensions on your character. A list of the first several penalties is given below.
\begin{itemize}
\item 1 weapon: no penalty
\item 2 weapons: -7 to attack checks 
\item 3 weapons: -15 to attack checks 
\item 4 weapons: -22 to attack checks 
\item 5 weapons: -37 to attack checks
\end{itemize}
You may invest in Multi-wield Skill to negate these penalties. Skill points in multi-wield can only negate the penalty, they cannot add to your attack check beyond that.

Putting a proficiency on multi-wield skill lets you treat the penalty as if you were wielding one fewer weapon than you are actually wielding. E.g. you can play a triple-sword wielding character who only needs to put 7 points in multi-wield skill, because he only takes the -7 penalty for double swords.

\section{Preparing an Action}
While we do progress in a linear fashion in-game, getting one character to perform all the actions they want to in a round before the next, the characters are not just standing still while they wait for everyone else to move. In actuality, initiative has given some characters priority to decide what their actions will be for the next 6 seconds before other characters decide on their actions. All the characters are performing their turns simultaneously, but we break up causality for convenience.

Because of this, you do not have to take all of your actions for the round when it is your turn. You can also prepare an action. Preparing an action lets you act based on things that another character does later in the round, or in the next round. You can, for instance, say, “Lyr prepares to dodge if Edrian attacks her.” Then if Edrian does in fact attack her on his next turn, Lyr will roll out of the way even though it is not her turn. You can also prepare to attack someone if they get in range, or leave cover. Preparing an action has some risk, because while it uses up one of the actions on your turn, it relies on some future condition that may not necessarily be true. However, used properly, preparing an action lends a lot of realism to a battle and makes combat very dynamic.

The number of conditions you may stack on a prepared action is equal to (Int/2) +1 rounded down (not towards zero). In other words, a character with an intelligence of 10 can only have one condition on their prepared action. For every 4 more intelligence a character has above 10, they may put one additional condition on a prepared action.

Characters with an intelligence of less than 10 may still put one condition on their prepared action, but they begin to take penalties to how many prepared actions they may take. A character with an intelligence of 9 or lower can prepare one less action than their total number of possible actions. A character with an intelligence of 5 or lower can only prepare half of their actions, rounded down. A character with an intelligence of 0 or 1 may not prepare any actions.

To give an example of a prepared action with multiple conditions, say Lyr has an intelligence of 14 (an Int of +2) and can therefore place 2 conditions on a prepared action. She may say “If Edrian attacks me, I will dodge, otherwise if Edrian leaves cover, I will attack him.” The number of ``If's'' you can put on a prepared action is the number of conditions. Note that in this case, the first condition filled already consumes the prepared action. If Edrian attacks, and then leaves cover, his attack consumes Lyr’s prepared action by making her dodge, and then he leaves cover without any retribution from Lyr.
More intelligent characters can place more conditions on their prepared actions to cover more possibilities without consuming more actions for preparation.

There are also two actions that you may specially prepare before a movement action, and remain prepared even after the movement action: Tank and Parry. Preparing to tank or parry a hit before a movement action can help deny any attacks that other characters have. You may prepare these before any skill or ability that causes movement.

\section{Special Actions}
Below is a list of all actions that have special effects during combat, and their descriptions.
\begin{description}
\item[Jump] [Str]- Jumping takes an action to perform, and two actions if you intend to perform a running start. A running start requires (3 -Str) squares to move, before performing the jump, excluding the square you are currently on. A minimum of 1 square is required to make a running start. A very strong character with a Str of 3 or more can therefore perform this with just one additional square.

You should set a goal height or goal distance before you roll, or specify that you are going to jump as far as possible. You may travel a number of spaces equal to your check horizontally and equal to half your check vertically. If you have a running start, you may travel a number of feet equal to your check horizontally, and equal to half your check vertically. These rules are reiterated in the combat section. A critically failed jump check may cause your character to faceplant and take damage. This is up to GM discretion. A horizontal jump requires a quarter of the distance of the jump as vertical clearance. In other words, every time you make a horizontal jump, you also make, you travel a quarter of the distance vertically.

If your character performs a jump where the sum of the horizontal distance and vertical distance they traverse is greater than 10, they must actually take standard falling damage upon both leaping and landing as if they have fallen a distance equal to the sum of their horizontal and vertical distance.
\item[Dodge] [Dex] - Dodging requires you prepare an action, and uses your tumble/roll skill.

Dodging can move you a distance equal to your character’s normal movement action. It requires that your dodge check be higher than the attacker’s attack check. For AOE damage sources that do not have an attack check, the character causing the damage may roll a simple Dex check (no attack bonus) vs your dodge to see if they successfully hit you before you dodge.

\item[Tumble/Rolling] [Dex] - Tumbling or rolling when you’re not dodging as a prepared action is also possible. It moves you one square less than your normal move action but temporarily substitutes your AC with your tumble check, and this is applied if any prepared attacks are triggered by the motion.

Tumbling/rolling as an unprepared action is also special because you are not allowed to prepare a parry before performing it like you can a regular move action.
\item[Tank/Block] [Con] - Tanking a hit must be done as a prepared action.

For every 10 you get on a tank check, you may remove one damage die from the source of damage. Your character may choose which die to remove. A critically successful Tank check negates all damage from the next damage source. A tank check of less than 5 causes your character to take one additional die of damage from the source, by rolling another die of the largest kind used in the damage calculation.

A critical failure on a tank check causes the damage source to roll for damage again, and your character will take both the initial and second rolls. Any successful Tank check also halves knockback. You may prepare to Tank a hit before a movement action.
\item[Parry] [Dex] - Parrying must be done as a prepared action. Parry is used to deflect a source of damage. You may not parry any area of effect sources of damage. Your parry check must exceed the check the damage source made to determine if it hit you.

Upon a parry that succeeds in excess of 5, you may choose to perform an attack check to try to redirect the damage to a specific target.

A failed parry check has no impact on incoming damage, but a critically failed parry check causes your character to take double damage.
\item[Grappling] [Str/Dex] - Grappling takes place in stages. It takes one action to initiate a grapple, and your grapple check must be higher than the Armor Class of the entity you are trying to grapple. The entity you are trying to grapple may then roll for grapple (using their preferred attribute) and if your roll is higher than their roll, a grapple is initiated. A critical success immediately initiates the grapple.

A grappled character may take an action to make a grapple check to escape the grapple. A critical success immediately escapes the grapple. A grappled character receives a penalty to attack checks equal to the difference between their last grapple check and the grappling character’s grapple last check.

The grappling character receives one free attack action of any kind on the grappled character at the beginning of each round. They also receive a bonus to attack checks on the grappled character equal to the difference between their last grapple check and the grappling character’s last grapple check.

The grappling character may also prepare grapple actions to use the grappled character as a meat shield for incoming damage. Any attack check less than the grapple check for these prepared action hits the grappled character instead of the grappling character.
\item[Feint/Juke] [Cha] - Feint may be used to trigger prepared actions that another character has made. Your feint check is rolled against the other character’s sense motive check.

If your Feint check exceeds their sense motive check, you may trigger one prepared action. For every 10 that your feint check exceeds their sense motive check by, you may trigger an additional prepared action that meets the same condition.

You must specify what your feint is appearing to do – if you are giving the impression your character is moving, or if you are giving the impression your character is attacking, or giving the impression that your character is performing some other action or skill. You will only set off prepared actions that are conditional based on if your character were actually performing the action they are pretending to perform.

If your feint check is successful, you receive one free action to do something else instead of the feint.

You may not prepare a feint action.
\end{description}