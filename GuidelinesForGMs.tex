\chapter{Guidelines For GM's}
\section{Awarding Levels}
Rp20 is meant to have characters level up very quickly, as characters may often need to invest in both class levels and standard levels. In our test games when developing the system, it hasn’t been uncommon to award our players 1-2 levels every session, just to ensure they are making progress towards their goals. It’s also best to start your characters with somewhere from 2-4 levels to invest, so that they may invest in special classes as they see fit, but still have enough things to work towards.
As a GM, you should be prepared to scale the difficulty appropriately as your characters level up rapidly.

There is also no codified experience system, but there are some ways of awarding levels that have worked well in our test games.
\begin{itemize}
\item Award everyone a level for completing a plot point
\item Optionally have everyone vote on an MVP, and award the MVP an additional level
\item Award levels for character growth.
\end{itemize}
Our GM’s have found that it keeps the players characters most engaged to reward them for causing progression of the story, and our players have found that the GM doesn’t feel as miffed about them not following the plot if the GM is invested in their character growth, rather than strictly following the story.
That brings us to another point. Rp20 indirectly enables you to keep the plot loose, which our GMs would highly recommend to keep players engaged. Define the world around your player’s characters, rather than forcing your characters into a world. This system; instead of having an exact codex of enemies, classes, weapons, etc.; simply has rules for how to create roughly balanced entities. You can use this to your advantage and bend upcoming plot points on the fly.

One of the other things you’ll notice about Rp20 is that it encourages heavy specialization- maximizing a specific set of skills at the expense of others, or “min-maxing” as it is known in the tabletop roleplaying community. There’s no cap on the amount of points you can put in a skill by default, and the fact that you can invest all of your levels in class levels and never gain any additional hitpoints or attribute points can have certain players build extremely fragile characters that are only very good at one particular thing.

The burden of mitigating minmaxing in Rp20 unfortunately falls largely to you, as the GM. Don’t be afraid to punish your characters for getting too specialized. Additionally, you’ll notice that the rules to Rp20 tend to be very far reaching, but very flexible. Your characters may very well come up with concepts that seem overpowered. Let them. This system is meant for it. Your defense against this should be to use what they come up with AGAINST them.
There are no standard antagonists or minions in Rp20, only what you come up with yourself. This is meant to let you play the antagonists intelligently, adapting to what the players come up with.

If your players come up with a net that teleports anyone encased in it into space, have your antagonist come up with a way to teleport back. Or even better, have a similar trap be used against them.

If your characters abuse styles like duplicitous or oblivious, pit them against foes that do the same.

\section{Difficulty Check Guidelines}
For the difficulty checks that aren’t written into the rules, use the following guidelines:
\begin{itemize}
\item Easy - 10
\item Standard - 15
\item Hard - 20
\item Very Difficult - 30
\item Near Impossible - 40
\item Only A Master of the Art has a chance - 50
\item Completely Impossible - 80
\end{itemize}
Because there’s no limit on how much characters may place in a skill, difficulty checks can have a wider range of numbers than many GM’s are used to. You should use a relatively wide range of numbers for difficulty checks, to encourage your players to try to get a character covering each situation.

\section{Custom Class Guidelines}
