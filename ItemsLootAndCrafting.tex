\chapter{Items, Loot, and Crafting}

All items have a craft grade which is meant to indicate that it is an average item for a character with a level equal to the craft grade.

All non-cosmetic items have a craft grade of 1 minimum.

This chapter needs to be rewritten into the correct sections instead of all being lumped into Base Craft Grade.

\section{Base Craft Grade}
The base craft grade outlines a few broad classes of items

\begin{itemize}
	\item Every +/-1 change to an attribute costs 2 craft grade
	\item Every +/-1 change to Reflex, Will, Fortitude, or Initiative costs 1 craft grade
	\item Every +2 to AC costs 1 craft grade
	\item Every +2 to a specific skill costs 1 craft grade
	
\end{itemize}

Typically increasing your own attributes increases the cost of an item. Decreasing your own attributes (taking a penalty) decreases the cost of an item by half the craft grade it would take to increase them by the same amount, for a maximum craft grade reduction of 3 grades. Items that apply a penalty to someone else AND items that apply a buff both increase the cost of an item.

Items that mimic the performance of a skill in any way (e.g. auto-parrying armor) cost 3 craft grade for each action that the character would otherwise have had to perform to perform the skill, and cost one craft grade for every 4 points on an effective roll that the item makes.

Items that mimic class abilities or produce class effects are dependent on the lowest number of levels a character would need to invest to produce that effect. If an item can produce multiple effects, but can only produce one at a time, you only need the lowest number of levels needed to produce the effect with the highest level. If an item can produce multiple effects at the same time, you must add the craft grades needed to produce each individual effect.

Melee weapons do damage including the dex and strength stats. Divide the average damage they do by dex and strength. Every 2 average damage per attribute modifier point is one additional craft grade. A melee weapon that is capable of doing more than 40 damage per attribute modifier point on an ideal roll is automatically legendary and cannot be crafted through normal means.

Ranged weapons do straight damage, and again craft grade is computed by average damage done. For every 8 average damage, it adds one additional craft grade. A ranged weapon may not be capable of dealing more than 40 damage per craft grade even if the average is below 8 per craft grade.

Ranged weapons by default have a falloff of 10, which increases penalty to hit by -2 for every 10 feet (5 spaces), and divides damage in half an additional time for every 10 spaces. To improve falloff by 10 feet, it costs an additional craft grade. Ranged weapons also have a standard cooldown of either ½ a round, or take two actions to reload. Improving the cooldown to the next geometric fraction ($\sfrac{1}{3}$, $\sfrac{1}{4}$, $\sfrac{1}{5}$) or reducing the number of actions needed to reload by one, costs one additional craft grade. You may also increase the cooldown by increments of half a round or increase the number of actions needed to reload by one to decrease the cost of a weapon by one craft grade for up to 3 craft grades of reduction. Burst fire weapons for game mechanical purposes are still treated as if they deal damage in a single instant.

Healing items require 1 craft grade for each 2d6 healing they provide.

AOE effects typically have a radius in which they have a full effect, and a radius double that in which they have a half effect. To make any effect AOE, add a craft grade for every 4 feet (2 spaces) of full effect radius.

Typically if a bonus is conditional upon something for its value, it still costs the number of craft grades you would need for the maximum possible bonus. A penalty only reduces the craft grade by the number of craft grades the minimum penalty would reduce the cost by. If the theoretical maximum bonus is infinite, it’s a legendary item. However depending on what the penalties and bonuses are conditional upon, you must further modify the number of craft grades with the following:

Making a bonus conditional upon a successful roll of any kind reduces the cost by 1 craft grade. Making a bonus conditional upon a failed roll of any kind increases the cost by 1 craft grade.

Making a penalty conditional upon a failed roll of any kind reduces the cost by 1 craft grade. Making a penalty conditional upon a successful roll increases the cost by one craft grade. These are not including the reduction to craft grade from the penalty itself.

A good way to figure out the value of currency in a game is to pin 100 of the currency (whatever currency is being used in the campaign) to 1 craft grade.

Self damage to reduce craft grade?

Conditional penalties and craft grade reduction?
conditionality based on your success, your failure, someone else's success against you, someone else's failure against you.

\section{Buffs}
Buffs add craft grades to the base craft grades.

\section{Penalties}
Penalties can subtract some craft grades from the base items.

\section{Conditionality}

\section{Examples}
Craft grades are probably the most complex set of rules in Rp20 when you first encounter them, so here are a few example items to demonstrate how craft grade is calculated.

\section{Crafting}
To craft an item, a Character must roll a craft check with at least $10 \times (Grade)$ to succeed.

Typically, an item of craft grade $n$ takes $n \times 8$ hours to craft. However, for every 10 on a craft check that a character rolls ABOVE the minimum craft grade, the amount of time needed to craft the item is cut in half.

\section{Loot Guidelines}
Average loot has a craft grade equal to the character level.

Really good loot might go up to 4 more craft grade than the current character level. It is not recommended to give characters more than 4 craft grades above the current character level.
