\chapter{Leveling Up}
This is where the extensibility of Rp20 comes into play.

Rp20 distinguishes between class levels and your \textsc{Character Level}.

Whenever you level up your character, you are either investing in a character level, or one of the special classes that your GM is using for the campaign. Typically, only gaining a \textsc{Character Level} gives you an additional point with which to improve a core textsc{Attributes}, and adds additional \textsc{Skill Points} for you to put into skills. The number of uses of your style is also dependent on \textsc{Character Level} alone.

So why would you ever spend a level on a special class? It \emph{is} entirely viable not to. However spending a level on a special class gives you access to abilities that you wouldn't have otherwise.

GM's may come up with skill trees and abilities that you may gain by investing a level in a special class, or use one of the example classes provided.

Balance can be tricky, which is why Rp20 tries to use levels as a sort of currency which helps to standardize what a character can gain by investing in a class level. This also makes it possible to mix and match classes from different settings without too much need to adjust for balance.

This is described further in the ``Guidelines for GM's'' chapter.

There are some base special classes built into Rp20, and they are listed below.

Just kidding! I still have to write this section.

