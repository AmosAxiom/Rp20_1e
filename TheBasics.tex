\chapter{The Basics}
Rp20, with a name slightly reminiscent of the well-weathered d20 system from Wizards of the Coast, is a role playing system based chiefly on the 20-sided die.

The main goal of any tabletop roleplaying game is really to have  a structured and codified way to make-believe. The chief way that this is accomplished in most systems is to quantify a character’s skill at a task, and then compare that skill against the difficulty of that task. Often, a dice roll will be added to the skill of the character to add an element of chance. In Rp20, characters have attributes such as strength, and whenever they have a task to accomplish based on strength, such as jumping over a wall, they roll a 20-sided die, and add their strength modifier (calculated from the strength attribute) to the value shown on the die. This sum is compared against what the GM (game master) determines to be the difficulty of that jump, and if the roll is higher than the difficulty, they succeed. There may be other numbers added to the roll, such as the character’s skill at jumping, or bonuses from equipment that aid jumping. This comparison between the ability to perform a task and the difficulty of the task is called a “check.” For instance, make a jump check to see if you make a jump.

Unless otherwise stated (in the rules, or by your GM) rolls of 1 and rolls of 20 have special significance. A roll of 1 is considered a ``critical failure.'' Typically this means that even if your check is high enough to succeed, you fail. A critical fail also may come with penalties to your character such as accidentally attacking themselves. Perhaps there’s an unlikely crack in the floor that your character steps in, which throws off their aim.

Rp20 is meant to have more simple rules than most, although more abstract and far-reaching. Many skills can be used in the place of many other skills, and there is no specific set of items that they are allowed to have- only rules on how to balance items. The purpose of this is to give new players an easy time understanding the system, but leave openness to the rules such that they allow for many creative combinations of character attributes.

Characters have many attributes, from the broadly applicable core attributes that determine how strong or how intelligent they are, to the very specific skills that they have trained. Rp20 also has some very broad attributes such as affinity, proficiencies, and style, which let your character bend chance in certain situations that they are meant to just be naturally good at.
You’ll notice that for many calculations, integer numbers are divided. Unless otherwise specified, upon dividing a number, round towards zero to the nearest integer.

\section{Core Attributes}
Rp20 uses the same core attributes found in many roleplaying systems: Strength, Constitution, Dexterity, Intelligence, Wisdom, Charisma. The corresponding modifiers for these are written Str, Con, Dex, Int, Wis, Cha. Three of these are physical– Str, Con, and Dex. The other three are mental– Int, Wis, and Cha.

In general, a perfectly average level 1 character would have 12 and 13 for most attributes, and 14 for one or two. 10 is the minimum you can have without taking minor penalties, and less than 8 in any attribute typically begins to accrue severe penalties as the game progresses. An attribute of 20 is considered incredibly good.

Having an attribute score of 0 for any attribute incapacitates your character, and may actually kill them if they have it for too long, based on GM discretion.

\begin{description}
\item[Strength] (Str) - used for tasks that require physical strength

Strength also determines the amount of damage you deal barehanded and with physical attacks, as well as how far your character can move with a single move action in combat. A normal character should have at least 8 Strength. Any less means that they find it difficult just to move their own body around.
\item[Constitution] (Con) - corresponds to the physical toughness of the character.

Constitution influences your number of hitpoints, changes how much damage you take, and determines how easily you can resist things like poison, disease, and the effects of strenuous labor.

A normal character should have at least 8 Constitution. Any less generally indicates that they are sickly or have something intrinsically wrong with their body that would cause them to die young even without exerting themselves.
\item[Dexterity] (Dex) - Besides corresponding to how dextrous your character may be, this is also used as a measure of your reflexes. As such it is used to determine who goes first when a fight starts, as well as if an attempt at sleight of hand or pickpocketing was successful, as well as if you hit your intended target.

A normal character should have at least 8 Dexterity. Any less than that means that they tend to trip over their own feet, and have trouble coordinating themselves.
\item [Intelligence] (Int) - Used for both what your character knows, as well as their ability to figure stuff out.

Commonly used for things like crafting, repair, and knowledge checks (knowlege checks are a great way to get the GM to tell you something you don't know). It also may be used for hacking checks or other things stereotypically attributed to “intelligence.” Intelligence also influences the number of points you can invest into skills.

A normal character should have at least 8 Intelligence. Any less means that they have trouble stringing together sentences, planning out the next few seconds of their life in their head, and in general making sense of the world.
\item[Wisdom] (Wis) - Used for your ability to perceive things, your intuition, and your willpower.

As a measure of your willpower, it is used to determine how easily you can fight off the influence of mind-altering substances or similar things. As a measure of your perception and intuition, it may be used to determine whether another character is bluffing, or gauge their true intentions. It is also a measure of your passive alertness to unusual or dangerous things while not directly looking for them. 

Wisdom also influences the number of actions you get during a round of combat.

A normal character should have at least 8 Wisdom. Any less than that means they have to exert tremendous mental effort just for basic awareness. A character with a Wisdom below 2 has trouble remembering to breathe.
\item[Charisma] (Cha) - Used as a measure of your personability.

This is used for diplomacy, bluffing, acting, animal taming, and other similar things that rely on your ability to hold other people hostage with pure charm.

Don't underestimate this mechanic, in the world of tabletop roleplaying, any lie you tell (“These aren’t the droids you’re looking for”) is a charisma-based bluff check weighted against a finite difficulty for the lie plus a wisdom-based sense-motive check on the part of whoever you're telling the lie too. With enough luck you should be able to convince the guard blocking the door that she's actually the color yellow, and should float off into the sky to rejoin the rest of the rainbow.

A normal character should have a Charisma of 8 or more. Any less and they are instantly perceived as unlikeable by everyone around them. Characters with Charisma below 4 intrinsically dislike themselves as well (although they don't have to be a low Charisma character to dislike themselves).
\end{description}

Numbers for these core attributes are typically assigned in one of the following ways at level 1:
\begin{enumerate}
\item Roll 8 20-sided die and drop the lowest 2 rolls ($\dicedrop{8}{20}{2}$). You may assign each roll to each attribute score however you wish. Your GM may still ask you to adjust these numbers for balance
\item Roll 6 20-sided die ($\dice{6}{20}$). If you have more than one roll that is less than 10, you may keep re-rolling the dice that are less than 10 until you only have one or no dice that show less than 10.
\item Allocate points to each core attribute until the sum of all core attributes is 85.
\item Repeatedly bribe your GM until they let you assign attributes as high as you like.
\end{enumerate}

Your GM will tell you by which method they prefer you to assign these attributes.

These numbers are modified as your character grows. Your character starts at level 1 with the numbers you have assigned. With every level your character gains beyond level 1, you may typically add one additional point to an attribute of your choice. Note that if your character invests a level into a specialty class, they may not receive this point to invest.

Rp20 makes this distinction between character levels and class levels. Whenever you level up, you may choose to level up your character, or invest it in a class that the GM is using for the campaign. In this system, it may actually be more general to say ``I have 8 levels'' than the usual ``I am level 8.''

\section{Calculated Attributes}
Some attributes are calculated from other attributes rather than being defined independently. 

For instance, for most situations, your core attributes are distilled into modifiers, numbers that are roughly proportional to the attribute. Those are used in checks, rather than using the core attributes directly. What follows is a list of these calculated attributes.

\begin{description}
	\item[Modifiers] - There is one modifier corresponding to each attribute. The modifier starts at 0 with 10 points in an attribute., and increases by 1 for every two additional points. Whenever you make a skill check relating to an attribute, you will roll a d20, then add your modifier, then add points you have in that skill and any bonuses. Another way of saying this is that the modifier is calculated by $(Attribute/2) - 5 $.
	\item[Hitpoints] - Your maximum hitpoints are calculated by $Level*(Con+1) + Constitution$. Note that both the Constitution attribute, and the Con modifier are used.
	
	The Level used here is the character level, and does not include levels you have invested in classes. Certain specialty classes may augment your maximum hitpoints in other ways.
	
	Having less than a -1 Con modifier will actually slowly decrease your health as you level up. This corresponds to a Constitution of 7 or less, indicating a sickly character.

	\item[Saving Throws] - There are some instances in which you would be paralyzed by a taser, get blinded by a flashbang, or be otherwise inconvenienced where quick reflexes, being tough or summoning proper willpower would save you from said inconvenience.
	
	Many weapons and environmental factors will have ways for you to avoid inconvenience by making a saving throw. A saving throw is just the pertinent attribute plus 1d20. Note that these are based on attributes, not the modifiers! There are three kinds of saving throws: reflex saves, will saves, and fortitude saves. These are standard to many roleplaying systems.
	\begin{description}
		\item[Reflex] saves are where quick thinking and movement gets you out of a situation, based on your Dexterity attribute. (The attribute, not the modifier!)
		\item[Will] saves are where willpower gets you out of a situation, based on your Wisdom attribute (not the modifier!). 
		\item[Fortitude] saves are where being physically robust gets you out of a situation, based on your Constitution attribute (not the modifier).
		These may all be modified by equipment or class characteristics.
	\end{description}
	
	\item[Armor Class] - Armor Class, or AC, is another calculated attribute standard to many roleplaying systems. Armor Class defines how difficult it is to land a damaging blow on your character.
	
	This may be because your character is covered in armor, but it may also be because your character is flexible and dodges easily.
	
	Armor Class by default is based on your Dexterity attribute, and adds bonuses to it based on how much armor you are wearing. Note that in Rp20, by default, for all Dex related skill checks you will take a penalty equal to the bonus your armor gives to your AC, unless you have invested in Armor Skill.
	
	\item[Skill Points] - All characters, at level 1, start with 10 skill points to invest. With character level you gain, you may add a number of points equal to your Int modifier. You gain a minimum of 1 skill point every level even if you have a 0 or negative Int modifier.
	
	For every class level you invest in, you also gain 1 skill point. Some classes may give you more skill points for investing a level in them.
\end{description}

Veteran roleplayers will notice that Rp20 minimizes randomness in these calculated attributes. The GM may still ask you to adjust them for balance regardless.

\section{Skills}
We've mentioned skills a few times now, without really going into how they work. Every character has a set of skills. If you’ve played other roleplaying systems, skills are pretty much done the standard way. The max number of points you can put into any skill is equal to twice your character level.

Veteran role-players will notice that Rp20 seems to encourage min-maxing to some extent. This is intentional. The system was built to encourage players and GM’s to break the game on each other. Skills are no different in this system.

All characters, at level 1, start with 10 skill points to invest. With each character level you gain, you may add a number of points equal to your Int modifier and distribute them among skills as you see fit.

You will add a minimum of 1 skill point each level, that is to say- characters with a negative Int modifier or a modifier of 0 still receive one point with each level gained. You may not remove skill points from a skill once you have invested in it, unless given explicit permission from the GM.

By default the limit to how many skill points you may put in a skill is twice your character level, but your GM may decide to limit the maximum number of points to your character level, or choose another limit. Note that you do not receive additional skill points if you invested your level in a class level, you only receive these points for investing in a character level.

There is technically no restriction on what kind of skill you can give your character points in, but they should not be too broad. This is up to GM discretion.

Common skills to put points into are listed below, with the corresponding modifier listed in square brackets. If a skill has an alternate name, it is listed as Name/Alternate Name. You need only list one of the names on your character sheet.

\begin{description}
\item[Sneak] [Dex] - Move silently and unnoticed by other people.
\item[Hide] [Dex] - Used to place an object in one spot unnoticed by other people. 
\item[Break/Sunder] [Str] - Used to break something, such as enemy weapons, enemy armor and the like.

For every 10 you get on your break check, you may break an item or weapon that is one additional craft grade higher than the weapon you are using. If you are trying to break something barehanded, your craft grade is considered equal to your character level (just the character level, not all of your levels together). A critical success on a break skill can let you break something up to 5 craft grades higher than normal.

A legendary item may only be broken by another legendary item with a critically successful break skill.

A critical failure on a break check will cause the item you are using to break instead of the item you are trying to break. If you are trying to break something barehanded and critically fail you will take (Craft grade)d6 damage, using the craft grade of the object you are trying to break. Craft grades are discussed further in the ``Items, Loot, and Crafting'' section.

\item[Tank/Block] [Con] - Used to shrug off damage done to your character. You must prepare this as an action before the damage occurs (described more in the ``Combat'' section).
For every 10 you get on a tank check, you may remove one damage die from the source of damage. Your character may choose which die to remove.

A critically successful Tank check negates all damage from the next damage source. A tank check of less than 5 causes your character to take one additional die of damage from the source, by rolling another die of the largest kind used in the damage calculation.

A critical failure on a tank check causes the damage source to roll for damage again, and your character will take both the initial and second rolls.

Any successful Tank check also halves knockback.
\item[Parry] [Dex] - Used to deflect a source of damage. You may not parry any area of effect sources of damage.

Your parry check must exceed the check the damage source made to determine if it hit you. Upon a parry that succeeds in excess of 5, you may choose to perform an attack check to try to redirect the damage to a specific target.

A failed parry check has no impact on incoming damage, and a critically failed parry check causes your character to take double damage.
\item[Feint/Juke] [Cha] - Feint may be used to trigger prepared actions that another character has made. Your feint check is rolled against the other character’s sense motive check.

If your Feint check exceeds their sense motive check, you may trigger one prepared action. For every 10 that your feint check exceeds their sense motive check by, you may trigger an additional prepared action that meets the same condition.

You must specify what your feint is appearing to do -- if you are giving the impression your character is moving, or if you are giving the impression your character is attacking, or giving the impression that your character is performing some other action or skill. You will only set off prepared actions that are conditional based on if your character were actually performing the action they are pretending to perform.
\item[Sleight of Hand/Pickpocket] [Dex] - Used for a variety of things. It is most commonly used to remove items from another character’s inventory without them noticing, or to place items within another characters inventory without them noticing.

Sleight of Hand allows you to cheat at games of chance, but also allows you to fool people about the location or condition of objects.
Any effect you’ve seen in any magic show is fair game, but many will be difficult to pull off.

Typically, for each object you are trying to mislead an observer about during the desired effect, you must roll a Sleight of Hand check, and they may make a separate Perception check for each of these objects. If they succeed at a single Perception check, you fail at producing the desired effect.
\item[Bluff] [Cha] - Used to convince another character of a falsehood. This can be a very powerful mechanic if the other character is unable to tell that they are being lied too.

Some GM’s treat Feint and Bluff interchangeably. Be sure to ask if this is the case.
\item[Diplomacy/Persuasion/Charm] [Cha] - Used to convince another character to perform an action or change their view of something. Can also be used to seduce or charm another character.
\item[Sense Motive] [Wis] - Used to tell if someone is lying or has an ulterior motive. Sense Motive is the counter-skill to the Bluff and Diplomacy skills.
\item[Perception] [Wis] - This is an ability to passively notice details or hidden things. Perception is one of the counter-skills to the Sneak and Hide skills.
\item[Search] [Wis] - This is an ability used to actively find details or hidden things. This is also one of the counter skills to the Sneak and Hide skills. Some GM’s may treat Perception and Search interchangeably. Be sure to ask your GM how this is handled.
\item[Tame/Handle Animal] [Cha] - This skill is used to get animals to do develop loyalty to you.
\item[Performance] [Cha] - Used to juggle, dance, sing, and act. The Performance skill is also used when you are trying to fake a situation, and convince onlookers that it is real. Some GM’s may actually treat Feint and Performance skills interchangeably. Be sure to ask your GM if this is the case.
\item[Disguise] [Cha] - Used to alter your appearance.
\item[Drive/Pilot] [Dex] - Used to handle a vehicle. This is also added to your attack check when trying to ram someone with a vehicle.
\item[Jump] [Str] - This skill is used to jump, both vertically, and across a horizontal gap. Since the jump skill is used for movement, there are a set of rules associated with it, described in more detail in the “Combat” section. In general, you may travel a horizontally a number of feet equal to half your roll, and vertically a number of feet equal to a quarter of your roll.
\item[Hold Breath/Abstain] [Con] - Used to hold your breath or abstain from another basic biological function (not eating, not urinating, etc.) for an extended period of time.
\item[Balance] [Dex] - Used to not fall down in tricky situations or small areas.
\item[Climb] [Str] - Used to scale a vertical or near-vertical slope.
\item[Tumble/Roll/Dodge] [Dex] - Used for several things. This skill can reduce falling damage, get behind opponents, and dodge damage. A tumble can automatically move you the equivalent of your character’s normal move distance.

When used to reduce falling damage, every 10 on your tumble check removes 1d6 from the falling damage roll. Normal falling damage is 1d6 for every 10 feet fallen.

\item[Swim] [Str] - Used to swim in a liquid medium against a current. In situations for characters that fly you may choose to use a ``Fly'' skill instead for something similar. 

The standard rules for swim and fly are that a character may move 2 spaces relative to the fluid for every 10 they get on a swim check.

\item[Navigate] [Wis] - Used to find your way to a desired location.

For every 10 on a navigate check, your GM will rule out 1 additional incorrect path to your intended destination. A critical fail provides no additional effect. A critical success instantly tells you the correct direction.

\item[Heal] [Wis] - Used to restore an organism's health.

For every 10 on your check, you may roll an additional d6 to determine how many hitpoints you restore. In other words, you heal $(Heal/10)d6$.

A heal check of less than 5 causes you to hurt the organism you are healing for 1 damage.

A critical fail on a heal check has the effect of repeating the last damage taken by the organism.

\item[Repair] [Int] - Used to repair a device or machine that has taken non-critical damage. For every 10 on your roll, you may repair an item of 2 additional craft grades.

\item[Disable Device] [Int] - This skill disables a machine or device. It can also be used to pick a lock. For every 10 on your roll, you may disable a machine of 2 additional craft grades.

Someone trying to later repair this device must exceed the roll you got to disable the device.

\item[Knowledge (History/Science/Engineering/Etc.)] [Int] - You must pick one specific category for each skill. Knowledge checks are used to find out if your character is aware of some fact about the world. They are an excellent way to get the GM to tell you something you don’t know.
\item[Craft (Device/Structure/Chemical/Etc.)] [Int] – You must pick one specific category for each skill. The category of the craft is honestly mostly cosmetic, all crafted items still follow the craft grade rules in terms of what attributes they can provide.
\item[Use Device] [Int] – Used to operate unfamiliar devices. This skill is also used to hack or find information through a computer.
\end{description}

\section{Special Skills}
These are skills that are treated differently from other skills, either they cannot specifically be rolled for, or they are used in very specific circumstances.

\begin{description}
\item[Armor Skill] (torso/limbs/head) - By default, if you’re wearing armor that gives you a +5 bonus to AC, you will take a -5 penalty to all Dex related skill checks. Armor Skill negates the penalty armor gives to Dex checks. However, it can only negate the penalty. Having more armor skill than you have armor bonus to your AC does NOT give you a bonus to all Dex related checks.

One additional restriction of armor skill is that you may not have more than 5 points more in one armor skill than you do in the next highest armor skill. That is to say, if you have 5 points in Armor Skill (torso), and wish to increase it, you must invest in either Armor Skill (head) or Armor Skill (limbs) in order to increase it further.

You may put a Proficiency on this skill (described in “Proficiency” section), which unlike most other proficiencies, doubles the contribution from skill points. In other words, if you are proficient in Armor Skill, and have 5 skill points in it, you can negate up to 10 armor penalty.

\item[Weapon Skill] (short-ranged/long-ranged, barehanded/melee/projectile/beam) - You must pick either a short-ranged or long-ranged weapon skill, and either a bare-handed, melee, projectile, or beam weapon skill. For example: Weapon Skill (long-ranged beam).

Weapon skill is added to your attack check. The more skill points you put in, the more likely it is you will land an attack with that weapon.

You may put a proficiency on this, but it does not change damage, and does not double your dex contribution to your hit check. A proficiency in a weapon skill only gives an additional +2 to hit, and prevents you from critically failing any attack checks with that weapon.

A few special cases are also listed below:
\begin{description}
\item[Weapon Skill] (long-ranged, barehanded) corresponds to your character’s ability to accurately throw things.
\item[Weapon Skill] (long-ranged, melee) corresponds to your characters ability to use a melee weapon that reaches more than 4 feet (2 squares) away from the square they are currently on. Any melee weapon with a reach shorter than that is considered short range.
\end{description}

\item[Multi-Wield Skill] - By default, wielding more than one weapon at the same time gives penalties to your attack check. The penalty starts at -7 for two weapons, goes up to -15 for three weapons, and wielding more weapons than that simultaneously takes a penalty equal to the previous two weapon numbers (e.g. -7-15 = -22 for four weapons, -15-22=-37 for five weapons, and so on. This is explained further in the Combat section). These penalties exist regardless of the number of limbs on your character.

Multi-wield skill negates these penalties. Skill points in multi-wield can only negate the penalty, they cannot add to your attack check beyond that.

Putting a proficiency on multi-wield skill lets you treat the penalty as if you were wielding one fewer weapon than you actually are. E.g. you can play a triple-sword wielding character who only needs to put 7 points in multi-wield skill that he is proficient in, because he only takes the -7 penalty for double swords.

\item[Grapple/Escape Artist] [Str/Dex] - Grapple is a special skill simply because it can use one of two attributes. Your character can choose to use whether to use Str or Dex at the time of using the skill, and the bonus from skill points applies regardless of which attribute is used.

The grapple skill can be used to disarm a character, both to initiate a grapple or a hold on another character, or to escape a grapple or hold that another character has on your character. Grappling is described further in the ``Combat'' section.

The grapple skill is actually also used when tying up or binding another character, and when trying to escape from binds that another character has placed on yours.
\end{description}

\section{Affinity}
Every character has an affinity for one of the core attributes. When you first create your character, you must choose which one your character has an affinity for. Your character can only have an affinity for a core attribute for which they have a score 14 or greater. However, the affinity need not be for your character’s highest score.

Whenever your character makes a check relating to the core attribute they have an affinity for, you may roll twice, and take whichever result they desire.

Note that for weapon damages and other effects that use attributes in their calculation, you may actually re-roll damage or effect results and pick the better of the two results.

Your character’s affinity may not change after creation, and every character may only have one affinity, unless they have attained godhood or some kind of similar state that generally removes that character from playability.

\section{Proficiencies}
Starting at level 3, a character may gain proficiency in one skill every 3 levels (level 3, 6, 9, 12, etc.). If you are proficient in a skill, it doubles the contribution of the core attribute to any check you perform for that skill.

For example, if a character is proficient in jumping with 5 skill points in jumping, and their Str modifier is 4, a jump check will be determined by 1d20 + 5 + 2*4, rather than 1d20 + 5 + 4.

You cannot critically fail a check for a skill you are proficient in. You do not re-roll, your check is what it is, but none of the severe penalties of a critical fail will occur. You can still store a roll of 1 as a critical fail for the Judicious style (described in the next section).

Special skills are treated differently if you put a proficiency on them. For instance, Armor Skill with a proficiency on it actually doubles the contribution of any skill points you put in it, since there is technically no associated attribute modifier used with this skill. The other special skills are described in the ``Special Skills'' section.

\section{Styles}
Styles refer to the way the universe treats your character in certain kinds of situations and the way your character performs. Styles may be social- referring to the kind of luck your character gets in social situations, intellectual- referring to the kind of luck your character gets when calculating or crafting things, or martial- referring to the kind of luck your character gets in combat and athletic situations. What situations a style applies to is up to GM discretion, and the terms are meant as guidelines.

Starting at level 1, a character may gain a style every 10 levels for up to three styles (level 1, 11, 21). You must not have two styles of the same type. For instance, if you already have a social style, you must obtain either an intellectual or martial style.

Unless otherwise specified, bonuses (and penalties) from these styles MUST be applied BEFORE the GM tells you if your check succeeded or not.

\begin{description}
\item[Adept] - If your character is adept, you may add a bonus of half your character level to all checks within the situation of the style.
\item[Systematic] - If your character is systematic, add a bonus of 1 to every check within the situation of the style. This bonus increases by 1 for every consecutive success until the bonus is equal to your character level, and then remains at your character level. If you fail a check, the bonus resets to 1.
\item[Lucky] - If your character is lucky, a number of times a day (in-game) you may flip a coin after rolling a die in a situation. If the coin returns heads, you may replace your roll with the highest possible number, counting as a critical success if used in a skill check. You may do this up to 7 times a day, or a number of times equal to your character level- whichever number is smaller.

You may use this ability multiple times for the same roll to try to obtain a critical success, so long as you do so before your GM tells you the outcome of the event. If you have poor luck with coin flips in real life, it is not recommended to choose a Lucky style for your character.
\item[Confident] - If your character is confident, once a day, upon making a check you may choose to re-use that roll for the next several consecutive checks in that situation, as if the die came up with the same number each time. The number of times you may reuse a roll is up to 10 times, or a number of times equal to your character level- whichever number is smaller.
\item[Judicious] - Whenever you roll a critical success or a critical failure in the situation of the style, you may choose to store the roll for future use, and re-roll the die for the current roll. If you have one or more stored rolls, you may discard your current roll to replace it with any of your stored rolls. You may store a number of rolls up to your character level. If you already have your maximum number of rolls stored, you may no longer store new rolls for later use.
\item[Awkward] - After the start of each in-game day, within a number of checks within the situation of the style up to 10 or your character level (whichever is less), you must roll a critical success. If you do not roll a critical success within this number of checks, every following check in that situation for the rest of the day is an automatic critical failure. However, if you roll a critical success within this number of checks, every following check for the rest of the day in that situation is a critical success.
\item[Duplicitous] - In any situation within the style, where you have a check opposing someone else’s (an NPC or a player character), you may steal their roll and replace it with yours, as if their die rolled your number and your die rolled their number. Note that this does not include bonuses that are added to the roll for the check. Every day, you may do this up to a number of times equal to your character level. You may only do this every other check at most, in other words, you may not use Duplicitous for two consecutive checks in the same situation.

For checks where the GM explicitly tells you whether you were successful or not, you may do this AFTER the GM tells you whether your check was successful or not. You may not use this style multiple times for the same check.

Duplicitous pairs well with characters reliant on skills such as bluff, perform, disguise, hide, sneak, grapple, and parry, as it enables you to negate the counter skills that other characters use on yours. A martially duplicitous character may also use the style exchange the die rolls from an initiative check.
\item[Inept] - If your character is inept, they take a penalty equal to their character level to every check within the situation of the style. However, they take a bonus of half of their character level for all other checks.
\item[Oblivious] - You may trigger the Oblivious style up to a number of times a day equal to your character level. If your character is oblivious, you may trigger oblivious by saying to the GM “what did you say?” or “what was that?” or any other request for repetition of a previous phrase while out of character. You may do this AFTER the GM has told you the outcome. This will ignore the current roll and force a re-roll.

You may reuse this style multiple times for the same check but it will burn additional uses of this style. Note that if your character is oblivious, even accidentally saying “what did you say” or equivalent phrases while out of character will trigger a re-roll and burn a use of this style.
\end{description}

To give an example of assigning a style, you may say, for instance, that your character is Martially Awkward. Then whenever your character is in combat, the rules of the Awkward style apply.